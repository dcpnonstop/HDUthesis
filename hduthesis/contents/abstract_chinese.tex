% !TEX root = ../main.tex

% 定义中文摘要和关键字
\begin{cabstract}
本文的目的是区分数码相机拍摄的自然图像(NI)和计算机图形渲染软件创建的计算机生成的图形(CG)。本文的主要贡献有三个。首先,我们提出利用两个不同的去噪滤波器来获取检测图像的一阶和二阶噪声,并假设残差噪声遵循所提出的统计模型来分析其特性。其次,在假设检验理论框架下,将NI与CG之间的识别问题顺利转移到似然比检验(LRT)的设计中,同时了解所有的干扰参数,同时从理论上考察LRT的性能。第三,在实际分类中,使用估计的模型参数,我们建议建立一个广义似然比检验(GLRT)。模拟和真实数据的大规模实验结果直接验证了我们提出的测试能够从NI中识别具有高检测性能的CG,并且显示出与一些现有技术相当的效果。此外,通过考虑一些后处理技术产生的攻击来验证所提出的分类器的鲁棒性。

\end{cabstract}

\ckeywords{自然图像,计算机生成的图形,数字图像取证,统计噪声模型,假设检验}
